\documentclass[uplatex,a4paper,twocolumn,twoside,10pt]{jsarticle}

\usepackage[top=20truemm,bottom=20truemm,left=25truemm,right=25truemm]{geometry}
\setlength{\columnsep}{8truemm}
\setlength{\footskip}{10truemm}
\setlength\intextsep{5pt}
\setlength\textfloatsep{0pt}

\usepackage{minted}
\usepackage{xcolor}
\usepackage{amsmath}
\usepackage{multirow}
\usepackage[dvipdfmx]{graphicx}
\usepackage{listings}
\usepackage{booktabs}
\usepackage{url}
\usepackage{lipsum}

\makeatletter
\def\@maketitle{% 
\begin{center}% 
{\LARGE \@title \par}% タイトル 
\end{center}% 
\begin{flushright}% 
{\large \@date\\}% 著者 
{\large \@author}% 日付
\end{flushright}% 
\par\vskip 1.5em
}
\makeatother

\begin{document}

\setlength{\baselineskip}{15pt}
\setcounter{page}{1}
%\hspace{10cm}
\title{予稿テンプレート例}
\author{○研究室 }
\date{2038/1/19(火)}
\maketitle

\section{はじめに}
ローカル環境を構築したい場合は,TexLiveをインストールするのが良いと思います.VSCodeの拡張機能を導入することで開発環境が快適になる上,コマンドの管理などの設定も楽になります.オンライン環境の場合はOverLeafというオンラインエディタがおすすめです.パッケージ管理や,環境変数の設定など面倒な作業が必要なくなります.さらにアカウントを作成すると,自動でバックアップが行われるためファイル管理がしやすいです.しかし無料版の場合,コンパイルに時間がかかるという大きな欠点があります.
以下\LaTeX の基本的な記法について解説します.

\section{セクション}
文書を論理的なブロックに分割するためにセクションを設けます.以下,セクションの例です.

\subsection{サブセクション}
サブセクションを使用して細かく分類することもできます.

\section{文章の書式}
LaTeXでは,文章の書式設定を行うためにコマンドや環境を使用します.以下,一般的な例です.

\subsection{強調}
文章中で特定の単語やフレーズを強調するには、\textbf{ボールド}などのスタイルを使います.

\subsection{箇条書き}
箇条書きを作成するためには、\texttt{itemize}環境や,\texttt{enumerate}環境を使います.
\begin{itemize}
  \item 項目1
  \item 項目2
  \item 項目3
\end{itemize}

\begin{enumerate}
\item \texttt{enumerate}環境
\begin{enumerate}
  \item 項目1
  \item 項目2
  \item 項目3
\end{enumerate}
\end{enumerate}


\subsection{数式}
数式は,\texttt{equation}環境や\texttt{\$...\$}のようなインライン数式で記述できます.

\begin{equation}
  E = mc^2
\end{equation}

数式のあとに\texttt{\textbackslash notag}を追加すると,数式の番号が消せます.
\begin{equation}
  1+2+3+4+\ldots = -\frac{1}{12} \notag
\end{equation}

以下のような複雑な数式の画像をLatexの形式に変換したい場合は,Mathpixというオンラインツール\url{https://mathpix.com/}がおすすめです.

\begin{equation}
\frac{1}{\pi}=\frac{2 \sqrt{2}}{99^2} \sum_{n=0}^{\infty} \frac{(4 n) !}{n !^4} \frac{26390 n+1103}{396^{4 n}}
\end{equation}

\texttt{texttt}を利用することでソースコードをうまく表現できます.\\
\begingroup
\footnotesize
\begin{equation}
\begin{split}
\texttt{\{ assign(N,D,S) : shift(S) \} = 1 :- day(D), nurse(N).}\notag
\end{split}
\end{equation}
\endgroup

\texttt{equation}内で,\texttt{split}を利用することで2段組で記述できます.\\
\begingroup
\footnotesize
\begin{equation}
\begin{split}
\texttt{\{assign(N,X,D):shift(X,\_,\_)\} = 1}\\
\texttt{:- day(D),nurse(N).} \notag
\end{split}
\end{equation}
\endgroup

\subsection{図}
図を挿入するためには,\texttt{figure}環境を使用します.図表にはキャプションを付けることもできます.図の場合は,キャプションを図の下に書きます.\texttt{[ width=0.3 \textbackslash textwidth ]}で画像サイズをあわせます.

\begin{figure}[ht]
  \centering
    \includegraphics[width=0.3\textwidth]{logo.png}
  \caption{図の例}
  \label{fig:example}
\end{figure}

\subsection{表}
表を作成するには,\texttt{tabular}環境を使用します.\texttt{booktabs}パッケージを使用すると,よりきれいになります.表の場合は,キャプションを表の上に書きます.表の例として,図の配置指定方法を示します.\texttt{\textbackslash begin{table}[ht]}のように指定してください.

\begin{table}[ht]
  \centering
  \caption{図の配置指定方法}
  \label{tab:sample}
  \begin{tabular}{llr}
    \toprule
    \textbf{指定文字} & \textbf{出力場所} \\
    \midrule
        h	& 記述した部分 \\
        t	& ページの上部 \\
        b	& ページの下部 \\
        p	& 独立したページ \\
    \bottomrule
  \end{tabular}
\end{table}

\section{Lorem Ipsum}
\lipsum[1-4]

\section{おわりに}
LaTeXの記法にはさまざまな機能がありますが,ここでは基本的な要素のみを紹介しました.参考文献は\texttt{bibliography}などを利用して記述します.別にbibのファイルで置く場合は以下のように記述します.卒業論文の場合などは体裁を整える必要があるので,そのままtexファイルに書いてしまうのが良いと思います.
\begin{minted}{tex}
\bibliographystyle{unsrt}
{\tiny
\bibliography{script}
}
\end{minted}

\begin{thebibliography}
\small

\bibitem{Lowry1951}
Lowry, H., Rosebrogh, N., Farr, A., Randall, R.: Protein measurement with the Folin phenol reagent, Journal of biological chemistry , Vol. 193, pp. 265-275 (1951).

\end{thebibliography}

\end{document}
